
\begin{thebibliography}{9}
\bibitem{Behr} 
 Behr, Merlyn J.; Wachsmuth, Ipke; Post, Thomas R.; Lesh, Richard. Order and Equivalence of Rational Numbers: A Clinical Teaching Experiment. Journal for Research in Mathematics Education, v. 15, n. 15, p. 323-341, 198 4.
  \bibitem{Confrey} Confrey, J.; Maloney, A.; Nguyen, K.; Mojica, G.; Myers, M. Equipartitioning/Splitting as A Foundation of Rational Number Reasoning using Learning Trajectories. Proceedings of the 33rd Conference of the International Group for the Psychology of Mathematics Education (p. 345–353). Thessaloniki, Greece, 2009.
  \bibitem{Cramer}   Cramer, Kathleen; Behr, Merlyn; Post, Thomas; Lesh, Richard. Rational Number Project: Initial Fraction Ideas. University of Minnesota, 2009.
   \bibitem{Empson}  Empson, Susan B. Equal Sharing and Shared Meaning: The Development of Fraction Concepts in A First-Grade Classroom. Cognition and Instruction, v. 17, n. 3, p. 283-342, 1999.
   \bibitem{Freitag}  Freitag, Mark A. Mathematics for Elementary School Teachers: A Process Approach. Cengale Learning, 2014.
    \bibitem{Garcez} Garcez, Wagner Rohr. Tópicos sobre O Ensino de Frações: Equivalência. Trabalho de Conclusão de Curso do PROFMAT, Instituto de Matemática Pura e Aplicada, 2013.
    \bibitem{IES} IES PRACTICE GUIDE WHAT WORKS CLEARINGHOUSE. Developing Effective Fractions Instruction for Kindergarten Through 8th Grade. Institute of Education Sciences, 2010. - Este relatório oferece um conjunto de diretivas, procedimentos e cuidados no ensino de frações nas séries iniciais que foram compilados a partir de relatos de experiência e estudos científicos.
    \bibitem{Lewin} Lewin, Renaio; López, Alejandro; Martínez, Salomé; Rojas, Daniela; Zanocco, Pierina. Números para Futuros Profesores de Educación Básica. ReFIP Matemática: Recursos para La Formación Inicial de Profesores Educación Básica. Ediciones SM Chile S.A., 2013.
    \bibitem{Litwiller} Litwiller, Bonnie H. Making Sense of Fractions, Ratios, and Proportions: 2002 Yearbook. National Council of Teache rs of Mathematics. 2002.
    \bibitem{Misconceptions} Mathematics Navigator. Misconceptions and Errors. America's Choice. Pearson, 2016.
    \bibitem{McNamara} McNamara, Julie; Shaughnessy, Meghan M. Beyond Pizzas and Pies, Grades 3-5, Second Edition: 10 Essential Strategies for Supporting Fraction Sense. Math Solutions Publications, 2015.
   \bibitem{MEC}  MEC, Brasil. Números Racionais: Conceito e Representação (TP6). Programa Gestão da Aprendizagem Escolar GESTAR I. 2007.
   \bibitem{Monteiro} Monteiro, Cecília; Pinto, Hélio. Desenvolvendo O Sentido de Número Racional. Associação de Professores de Matemática, 2009.
    \bibitem{Musser} Musser, Gary L.; Peterson, Blacke E.; Burger, William F. Mathematics for Elementary Teachers: A Contemporary Approach. John Wiley $\&$ Sons, Inc., 2014.
    \bibitem{Ni} Ni, Yujing. How Valid is It To Use Number Lines to Measure Children's Conceptual Knowledge about Rational Number? Educational Psychology: An International Journal of Experimental Educational Psychology, v. 20, n. 2, p. 139-152, 2000.
    \bibitem{Pearn} Pearn, Catherine; Stephens, Max. Why You Have To Probe To Discover What Year 8 Students Really Think about Fractions. MERGA 27, v. 2, p. 430-437, 2004.
    \bibitem{Pereira} Pereira, Ana Paula Cabral Couto. O Ensino de Frações na Escola Básica: O Currículo Common Core nos EUA, Hung-Hsi Wu e Uma Análise Comparativa em Dois Livros Didáticos do PNLD. Dissertação (Mestrado Profissional em Rede Nacional PROFMAT), Universidade Federal Fluminense, 2015.
   \bibitem{Petit}  Petit, Marjorie M.; Laird, Robert E.; Marsden, Edwin L. A focus On Fractions: Bringing Research To The Classroom. Studies in Mathematical Thinking and Learning. Taylor $\&$ Francis, 2010.
   \bibitem{Post} Post, Thomas R.; Wachsmuth, Ipke; Lesh, Richard; Behr, Merlyn J. Order and Equivalence of Rational Number: A Cognitive Analysis. Journal for Research in Mathematics Education v. 16, v. 1, p. 18-36, 1985.
   \bibitem{Pothier} Pothier, Yvone; Sawada, Daiyo. Partitioning: The Emergence of Rational Number Ideas in Young Children. Journal for Research in Mathematics Education, v. 14, n. 4, p. 307-317, 1983.
   \bibitem{Schliemann}  Schliemann, Analúcia; Carraher, David W.; Caddle, Mary C. From Seeing Points To Seeing Intervals in Number Lines in Graphs. Em: Brizuela, Bárbara M.; Gravel, Brian E. (Ed.). Show Me What You Know, Teachers College, Columbia University, 2013.
   \bibitem{Small} Small, Marian. Uncomplicating FRACTIONS To Meet Common Core Standards in Math, K-7. Teachers College Press, 2013.
   \bibitem{Spangler} Spangler, David B. Strategies for Teaching Fractions: Using Error Analysis for Intervention and Assessment. Corwin, 2011.
    \bibitem{Tierney} Tierney, Cornelia; Berle-Carman, Mary. Fractions: Fair Shares. Investigations in Number, Data, and Space. Dale Seymour Publications, 1998.
    \bibitem{Tierney2} Tierney, Cornelia. Different Shapes, Equal Pieces: Fractions and Area. Investigations in Number, Data, and Space. Scott Foresman, 2004.
    \bibitem{Thomaidis} Thomaidis, Yannis; Tzanakis, Constantinos. The Notion of Historical “parallelism” Revisited: Historical Evolution and Students' Conception of The Order Relation On The Number Line. Educational Studies in Mathematics, v. 66, p. 165-183, 2007.
    \bibitem{Vamvakoussi}  Vamvakoussi, Xenia; Vosniadou,Stella. Bridging the Gap Between the Dense and the Discrete: The Number Line and the “Rubber Line” Bridging Analogy. Mathematical Thinking and Learning, v. 14, n. 4, p. 265-284, 2012. DOI: 10.1080/10986065.2012.717378. - Este artigo faz um resumo histórico de como a noção de densidade surgiu.
   \bibitem{Vance} Vance, James H. Understanding Equivalence: A Number by Any Other Name. School Science and Mathematics, v. 92, n. 5, p. 263-266, 1992.
   \bibitem{Walle} Van de Walle, John A. Matemática no Ensino Fundamental. Formação de Professores e Aplicação em Sala de Aula. Sexta edição. Artmed, 2009.
   \bibitem{Ventura} Ventura, Hélia Margarida Gaspar Lopes. A Aprendizagem de Números Racionais através das Conexões entre As Suas Representações: Uma Experiência de Ensino no 2.o Ciclo do Ensino Básico. Tese de doutorado, Instituto de Educação, Universidade de Lisboa, 2013.
    \bibitem{Wu} Wu, Hung-Hsi. Understanding Numbers in Elementary School Mathematics. American Mathematical Society, 2011.
\end{thebibliography}
