\chapter{Introdução}


Frações é certamente um dos tópicos que mais desafia o ensino e a aprendizagem de matemática na Educação Básica. Justamente por isso, tanto se publicou sobre o assunto nas últimas décadas (para citar apenas algumas das mais utilizadas:  {\it Rational Number Project, Institute of Education Science} (\cite{IES}, 2010), Van de Walle (\cite{Walle}, 2009) e Wu (\cite{Wu}, 2011). Este texto, organizado como uma proposta didática,  reúne as reflexões e as discussões dos autores sobre o tema, amparadas por essas publicações e pela análise de livros didáticos de diversos países. A proposta aqui apresentada foi planejada para:

\begin{enumerate}[(i)]
\item  ser aplicada diretamente em sala de aula, como material didático destinado aos anos intermediários do ensino fundamental (do $4^{\textrm{\underline{o}}}$ ao $7^{\textrm{\underline{o}}}$ ano) e
\item amparar a formação e o desenvolvimento profissional do professor que ensina matemática na educação básica.
\end{enumerate}

O texto concentra-se na abordagem inicial de frações como objeto matemático, ou seja, como novas quantidades. Busca-se, assim, aumentar o universo numérico do estudante, explorando o assunto a partir de atividades que visam à construção conceitual do tema e a conduzir os alunos a desenvolverem o raciocínio matemático amparados por reflexão e por discussão. Assim, as atividades visam a desafiar os alunos e a levá-los a estabelecer suas próprias conclusões sobre os assuntos tratados. Busca-se valorizar a capacidade cognitiva dos alunos, respeitando uma organização crescente e articulada de dificuldade na organização das atividades. Espera-se com isso mudar a perspectiva do binômio quantidade/qualidade. No lugar de uma quantidade enorme de exercícios, são propostas poucas  atividades que exigem maior reflexão e aprofundamento dos conceitos. Assim, são evitadas atividades de simples observação e repetição de modelos e os tradicionais ``exercícios de fixação'', que, pontuais, são apenas com o objetivo de desenvolver a fluência em procedimentos específicos (por exemplo, os que envolvem a equivalência entre frações).

Uma outra característica particular deste material é o diálogo com o professor. No início de cada lição, há uma introdução dirigida especificamente ao professor que apresenta os objetivos da lição, uma discussão dos aspectos matemáticos que serão tratados, as dificuldades esperadas e algumas observações sobre os passos cognitivos envolvidos. Diferente dos livros didáticos tradicionais, em que, para o professor, há pequenas observações pontuais junto ao texto do aluno e um longo texto teórico anexo ao final do livro, nesta proposta a ``conversa'' com o professor é permanente. Em cada atividade são realizadas discussões sobre os objetivos a serem alcançados, recomendações e sugestões metodológicas para sua execução e, quando pertinente, uma discussão sobre algum desdobramento do assunto tratado.

% Incluir aqui as descrições das seções de cada lição (Explorando o Assunto, Organizando as Ideias, Mão na Massa, Quebrando a Cuca e Refletindo).

Entende-se que, nesta etapa da escolaridade, considerando o cotidiano próprio do aluno, o conceito de fração aparece ligado a  noções informais traduzidas por expressões como metade, terço, quartos, décimos e centésimos, por exemplo. Assim, nas primeiras duas lições, buscou-se utilizar a linguagem verbal e os conhecimentos anteriores dos estudantes sobre situações em que aquelas expressões são utilizadas para conduzir as primeiras abordagens, visando à introdução de um conhecimento mais organizado e formal sobre o assunto. Apenas posteriormente, são introduzidas a linguagem e a simbologia próprias da matemática.

%Rever à luz do arquivo do GDocs
A introdução das frações na Educação Básica amplia o universo numérico do aluno e envolve um salto cognitivo, ir além da contagem. São duas as principais questões nesse processo: ``a identificação de uma unidade não explícita \textit{a priori}'' e a compreensão de uma  ``unidade contínua'', isto é, que pode ser subdividida em qualquer número de partes.

A construção de ideias abstratas, especialmente nesta etapa da escolaridade, deve ser amparada por contextos e modelos representativos. Na abordagem aqui proposta decidimos por iniciar apenas a partir de situações que envolvem modelos contínuos (linhas e regiões do plano ou do espaço). Assim, por exemplo, não trataremos de ``um terço de uma caixa de lápis'', mas de ``um terço de uma barra de chocolate''. 

A decisão por evitar modelos discretos em um momento inicial deve-se aos seguintes fatos: (i) modelos discretos já evidenciam uma unidade a priori; por exemplo, na determinação de um terço de 24 lápis, a unidade ``lápis'' não é nem a unidade nem a subunidade que precisam ser levadas em conta para a determinação da fração ``um terço de 24 lápis'' e (ii) como o conceito de fração subentende o de equipartição, contextos discretos podem desencadear discussões mais complexas, por exemplo, o que seria determinar $1/10$ de uma caixa de 24 lápis?

A opção por modelos contínuos traz limitações inerentes. É natural que os estudantes associem a fração à forma que a identifica no modelo. É necessário que identifique-se a fração não à forma, mas à quantidade evidenciada na representação. Assim, por exemplo, se o modelo for um retângulo, o que está em questão é a área e a fração $1/4$ pode ser representada igualmente por um retângulo ou um triângulo, como na figura a seguir (ver Atividade 4 da Lição 1).

Iniciar o estudo de frações a partir de modelos contínuos é uma decisão compartilhada por propostas que caracterizam livros japoneses e franceses.

\begin{center}
  \begin{tikzpicture}[scale=5]
  \draw[fill=common, fill opacity=.3] (15,0) rectangle (19,3);
  \draw[fill=common, fill opacity=.3] (15,3) rectangle (19,6);
  \draw (15,1.5) -- (19,1.5);
  \draw (15,3) -- (19,6);    
  \end{tikzpicture}
\end{center}

As lições 1 e 2 introduzem os conceitos elementares e a linguagem de frações a partir de situações concretas e de modelos contínuos. Na Lição 1, as frações emergem de situações concretas amparadas pela linguagem verbal. Uma vez estabelecida a unidade, a expressão ``fração unitária'' nomeia cada uma das partes da divisão da unidade em partes iguais. Nas atividades dessas lições a unidade está fortemente vinculada a um objeto concreto. Assim, por exemplo, a fração de uma torta, não é ainda tratada com a abstração própria do conceito de número, mas como uma fatia da torta em uma equipartição. Toma-se bastante cuidado com o papel da determinação da unidade e com a necessidade de uma ``equipartição'' para a identificação de uma fração. A notação simbólica de frações e as frações não unitárias, incusive as maiores do que a unidade, surgem apenas na Lição 2. As frações com numerador diferente de 1 são apresentadas a partir da justaposição de frações unitárias com mesmo denominador ou simplesmente contando-se essas frações. Para isso, tem-se a representação pictórica como um apoio importante. Nessas lições, as atividades são quase majoritariamente para identificar, reconhecer, analisar e justificar.

Na Lição 3, é exigida maior abstração dos alunos. Retoma-se a representação de números na reta numérica, enfatizando-se, no contexto das frações, a associação do segmento unitário à unidade. Os modelos contínuos e a justaposição de partes correspondentes às frações unitárias são a base da proposta desenvolvida. A representação das frações na reta numérica é usada para amparar a abordagem da comparação de frações com um mesmo numerador e com um mesmo denominador. Além disso, são propostas atividades que tratam a comparação de frações a partir de uma referência.

A Lição 4 trata da equivalência de frações tendo como objetivo a sua função na comparação de duas frações quaisquer. O assunto é abordado utilizando-se representações equivalentes em modelos de área retangulares, em modelos de área circulares e na reta numérica. A inclusão de modelos diferentes é proposital pois, com isso, o aluno tem a oportunidade de perceber as mesmas propriedades em contextos diferentes. Finalizando a lição, são propostas atividades que conduzem à exploração da propriedade das frações que garante que, dadas duas frações diferentes, é sempre possível determinar uma terceira fração que está entre elas (propriedade de densidade).

Adição e subtração de frações são o tema da Lição 5.
A abordagem dessas operações será a partir de problemas e fundamentada na equivalência de frações, que permite determinar subdivisões comuns da unidade para expressar as frações envolvidas nos cálculos.
Os significados e os contextos que caracterizam as operações de adição e de subtração ​envolvendo frações são semelhantes àqueles que compõem a abordagem dessas operações com números naturais, ​o que ​promove​​ uma continuidade conceitual ​no desenvolvimento desse assunto.

Este volume marca o início de um trabalho em desenvolvimento, que será ampliado e complementado por novos volumes e novas edições. Para o volume 2, de mesmo tema, está prevista a complementação da abordagem das operações com frações, trazendo a multiplicação e a divisão envolvendo frações, a abordagem de frações em situações e modelos discretos e o uso de frações em contextos de razão e de proporção, além das porcentagens.

Teremos prazer em considerar suas sugestões para este livro por meio do endereço eletrônico \textcolor{blue}{\url{livroaberto@impa.br}}.
A edição mais recente deste livro está disponível em \textcolor{blue}{\url{umlivroaberto.org}}.
